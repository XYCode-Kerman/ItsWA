\documentclass[UTF8]{ctexart}
\usepackage[scale=0.8]{geometry}
\usepackage{ulem, paralist}

\newcommand{\itswa}{ItsWA}

\title{\textbf{\itswa 用户手册}}
\author{XYCode}
\date{编译日期:\today}

\begin{document}
    \maketitle

    \tableofcontents

    \section*{关于本文}
        本文档是Github项目 \itswa 的用户手册。

        \itswa 是一个基于Python搭建,使用Lrun提供安全运行时的Linux下的竞赛代码评测系统。
    
    \section{基本架构}
        本项目主要由三部分组成。

        \subsection{管理器 Manager(WIP)}
            管理器用于提供一系列TUI、GUI供用户调用,以实现对比赛和评测的管理。

        \subsection{评测器 Judge}
            评测器用于评测提交者的代码,评测器支持使用多线程评测,以实现大规模快速评测。

            本项目的评测器在选手代码编译前会对代码中的\textbf{危险系统调用}进行过滤,并且使用了Lrun作为\textbf{安全运行时},实现了在\textbf{syscall}级的安全防护。

        \subsection{在线评测系统 Online Judge(WIP)}
            本项目自带一个在线评测系统,选手可直接在该平台内提交自己的代码并获得即时评测结果。由于本项目的评测器存在被卡的可能性,因此\textbf{本项目自带的OJ \uline{不得被用于搭建OJ平台}},我们仅推荐将其作为校级、市级等小型非正式比赛的一个提交通道。
    
    \section{系统要求}
        注:\itswa 的开发环境为 ArchWSL,如果您在其他平台上遇到了问题,请及时通过Issue向我们报告。

        \subsection{操作系统及软件运行时}
            操作系统:Linux 任意发行版,正式评测请使用\textbf{NOI Linux}。

            软件运行时:Python $\geq$ 3.10

        \subsection{编译器}
            \begin{tabular}{|c|c|}
                \hline
                \textbf{语言} & \textbf{命令(确保其可通过Shell调用)} \\
                \hline
                对于C++ & g++ (x86\_64-pc-linux-gnu) \\
                \hline
                对于C & gcc \space \space (x86\_64-pc-linux-gnu) \\
                \hline
            \end{tabular}

        \subsection{安全运行时}
            libseccomp = 2.x。

            Linux $\ge 2.6.26$ 最低,$\ge 3.12$ 推荐。

            Lrun 最新版本。
        
        \subsection{NOI Linux}
            您可以只需在NOI Linux上安装\textbf{安全运行时},即可使用本软件。
    
    \section{比赛配置文件 Contest Config File}
        比赛配置文件(Contest Config File)简称CCF\footnote{你知道为什么这个文件不叫作 Contest Profile。},是存在于比赛目录下的一个 Json 配置文件。

        \itswa 使用Pydantic来读取和解析CCF,\textbf{如果您在使用过程中出现了有关 Pydantic 的报错,那么\uline{很有可能是CCF的格式不正确}。}
        
        一些无法使用 Json 存储的数据结构被称作 Binary Config File,简称 BCF,它们将会被保存在比赛目录下的 bcf/ 目录。

        我们非常\textbf{不推荐}您手动编辑CCF,因为这可能导致错误,您应当使用 Manager 来编辑它们。如果您希望手动编辑CCF,请查看 \itswa 的CCF解析器中的定义\footnote{基于Pydantic定义的数据结构类型十分直观,因此我们认为\itswa CCF 解析器的中的定义应当成为 CCF 的事实标准。}后再编辑。
\end{document}