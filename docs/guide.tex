\documentclass[UTF8]{ctexart}
\usepackage[scale=0.8]{geometry}
\usepackage{ulem, paralist}

\newcommand{\itswa}{ItsWA}

\title{\textbf{\itswa 用户手册}}
\author{XYCode}
\date{编译日期:\today}

\begin{document}
    \maketitle

    \tableofcontents

    \section*{关于本文}
        本文档是Github项目 \itswa 的用户手册。

        \itswa 是一个基于Python搭建,使用Lrun提供安全运行时的Linux下的竞赛代码评测系统。
    
    \section{基本架构}
        本项目主要由三部分组成。

        \subsection{管理器 Manager(WIP)}
            管理器用于提供一系列TUI、GUI供用户调用,以实现对比赛和评测的管理。

        \subsection{评测器 Judge}
            评测器用于评测提交者的代码,评测器支持使用多线程评测,以实现大规模快速评测。

            本项目的评测器在选手代码编译前会对代码中的\textbf{危险系统调用}进行过滤,并且使用了Lrun作为\textbf{安全运行时},实现了在\textbf{syscall}级的安全防护。

        \subsection{在线评测系统 Online Judge(WIP)}
            本项目自带一个在线评测系统,选手可直接在该平台内提交自己的代码并获得即时评测结果。由于本项目的评测器存在被卡的可能性,因此\textbf{本项目自带的OJ \uline{不得被用于搭建OJ平台}},我们仅推荐将其作为校级、市级等小型非正式比赛的一个提交通道。
    
    \section{系统要求}
        注:\itswa 的开发环境为 ArchWSL,如果您在其他平台上遇到了问题,请及时通过Issue向我们报告。

        \subsection{操作系统及软件运行时}
            操作系统:Linux 任意发行版,正式评测请使用\textbf{NOI Linux}。

            软件运行时:Python $\geq$ 3.10

        \subsection{编译器}
            \begin{tabular}{|c|c|}
                \hline
                \textbf{语言} & \textbf{命令(确保其可通过Shell调用)} \\
                \hline
                对于C++ & g++ (x86\_64-pc-linux-gnu) \\
                \hline
                对于C & gcc \space \space (x86\_64-pc-linux-gnu) \\
                \hline
            \end{tabular}

        \subsection{安全运行时}
            libseccomp = 2.x。

            Linux $\ge 2.6.26$ 最低,$\ge 3.12$ 推荐。

            Lrun 最新版本。
        
        \subsection{NOI Linux}
            您可以只需在NOI Linux上安装\textbf{安全运行时},即可使用本软件。
    
    \section{比赛配置文件 Contest Config File}
        比赛配置文件(Contest Config File)简称CCF\footnote{你知道为什么这个文件不叫作 Contest Profile。},是存在于比赛目录下的一个 Json 配置文件。

        \itswa 使用Pydantic来读取和解析CCF,\textbf{如果您在使用过程中出现了有关 Pydantic 的报错,那么\uline{很有可能是CCF的格式不正确}。}
        
        一些无法使用 Json 存储的数据结构被称作 Binary Config File,简称 BCF,它们将会被保存在比赛目录下的 bcf/ 目录。

        我们非常\textbf{不推荐}您手动编辑CCF,因为这可能导致错误,您应当使用 Manager 来编辑它们。如果您希望手动编辑CCF,请查看 \itswa 的CCF解析器中的定义\footnote{基于Pydantic定义的数据结构类型十分直观,因此我们认为\itswa CCF 解析器的中的定义应当成为 CCF 的事实标准。}后再编辑。
    
    \section{版权声明}
        \subsection{声明}
            本文件是 \itswa 的一部分。

            \itswa 是自由软件:你可以再分发之和/或依照由自由软件基金会发布的 GNU 通用公共许可证修改之,无论是版本 3 许可证,还是(按你的决定)任何以后版都可以。
            
            发布 \itswa 是希望它能有用,但是并无保障;甚至连可销售和符合某个特定的目的都不保证。请参看 GNU 通用公共许可证,了解详情。
            
            程序中附带一份许可证副本。如果没有,请看 https://www.gnu.org/licenses/。

            \textbf{GNU 通用公共许可证是\uline{传染性协议},这意味着所有使用到 \itswa 或基于 \itswa 修改的程序应当使用\uline{相同的}许可证并开放源代码。}

            \textbf{但是根据本软件作者的意愿,如果您有修改本软件并闭源的需求,请依照下方的模板,发送一份申请到 xycode-xyc@outlook.com,获得许可后您可无视GNU 通用公共许可证中的条款来对本软件的某一版本进行修改和再分发\footnote{作者的“授权”只对申请的版本有效,而不针对本软件未来将会发布的版本。}。}

        \subsection{申请模板}
            我方因 \textit{此处填写原因} 有对 \itswa 软件的 \textit{此处填写版本(git hash 或版本号)} 版本进行\textbf{修改}后闭源并\textbf{再分发}的需求,

            但因为 GNU 通用公共许可证中的条款\textit{此处填写条款序号\footnote{示例:第x章第x款}} 的限制,

            因此根据 \itswa 软件用户手册中 \textit{根据本软件作者的意愿,如果您有修改本软件并闭源的需求,请依照下方的模板,发送一份申请到 xycode-xyc@outlook.com,获得许可后您可无视GNU 通用公共许可证中的条款来对本软件的某一版本进行修改和再分发} 的声明,发送本份邮件以获取对 \itswa 软件该版本的修改及再分发的授权。

            \rightline{
                \textbf{申请人名称(公司、社会团体或个人)}
            }
        
        \subsection{授权模板}
            \noindent 尊敬的 \textbf{申请人名称(公司、社会团体或个人)}:
            
            我方在接收并阅读你方的申请后,决定依照用户手册中的内容,对你方免除GNU 通用许可证中的\textit{条款内容},即\textit{条款内容}对你方不适用。

            注意:该授权仅对 \itswa 的 \textit{版本} 有效,对 \itswa 的其他版本无效。

            \rightline{\textbf{\itswa 软件作者 XYCode Kerman \footnote{注:因某些原因,我不方便公开我的真实姓名,但我的网名 XYCode Kerman 的签名任然应当具有和真实姓名签名相等的法律效力。}}}
            \rightline{\textit{作者签名图片}}
\end{document}